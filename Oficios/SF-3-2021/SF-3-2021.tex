%----------------------------------------------------------------------------------------
%	PACKAGES AND OTHER DOCUMENT CONFIGURATIONS
%----------------------------------------------------------------------------------------

\documentclass[12pt]{article}
\usepackage{alltt}
\usepackage[spanish]{babel}
%\usepackage[extreme]{savetrees}
\usepackage[
   % showframe,
    paper=letterpaper,
    headheight=54.45pt,
    bottom=1.2in
]{geometry}
\usepackage[utf8]{inputenc}
\usepackage{fancyhdr} % Required for custom headers
\usepackage{graphicx}
\usepackage{lastpage} % Required to determine the last page for the footer
\usepackage{setspace}
\usepackage{hyperref}
\usepackage[theoremfont,largesc,tighter,osf]{newpxtext}
\usepackage{xcolor}

%%%%%%%%% === Document Configuration === %%%%%%%%%%%%%%

\pagestyle{fancy}
\lhead{
\includegraphics[width=0.2\textwidth]{../../Logos/FEUCR.pdf}
} % Top left header
\chead{} % Top center header
\rhead{
\includegraphics[clip, trim=1.3cm 16.3cm 11.85cm 0.5cm, width=0.07\textwidth]{../../Logos/SF.pdf}
} % Top right header
\lfoot{} % Bottom left footer
\cfoot{Secretaría de Finanzas, FEUCR.\\
        Ciudad Universitaria Rodrigo Facio,
        costado oeste del comedor estudiantil.\\
        {\bfseries Teléfono:} (506) 2511-4615\quad
        {\bfseries E-mail:} \texttt{finanzas.feucr@ucr.ac.cr}
       } % Bottom center footer
\rfoot{P\'ag.\ \thepage\ de\ \pageref{LastPage}
       } % Bottom right footer

\renewcommand{\footrulewidth}{0.3pt}% default is 0pt
\renewcommand{\headrulewidth}{0.3pt}% default is 0pt
\def\baselinestretch{1.5}% Interlineado
\setlength{\parindent}{0.09\linewidth}% Sangria

\newskip\smallskipamount \smallskipamount=6pt plus 2pt minus 2pt
\newskip\medskipamount   \medskipamount  =12pt plus 4pt minus 4pt
\newskip\bigskipamount   \bigskipamount =18pt plus 6pt minus 6pt

\newcommand{\MONTH}{%
  \ifcase\the\month
  \or enero% 1
  \or febrero% 2
  \or marzo% 3
  \or abril% 4
  \or mayo% 5
  \or junio% 6
  \or julio% 7
  \or agosto% 8
  \or septiembre% 9
  \or octubre% 10
  \or noviembre% 11
  \or diciembre% 12
  \fi}

%----------------------------------------------------------------------------------------
%	ARTICLE CONTENTS
%----------------------------------------------------------------------------------------
\begin{document}


\begin{flushright}
  %\textcolor{white}{\textbf{CONSECUTIVO}}\\
  SF-3-2021\\
  \medskip
  \the\day\ de \MONTH, \the\year
\end{flushright}
\medskip
\begin{flushleft}\begin{spacing}{1}
  \textbf{Srta. Kenya Porras Salas\\
  Tesorería\\
  Asociación de Estudiantes de Estudios Generales}
\end{spacing}\end{flushleft}

\noindent Estimada tesorera:\par

Cordial saludo de mi parte.\par
El procedimiento que busca realizar se conoce como \emph{solicitud de inclusión} al catálogo. Esta solicitud se puede realizar siempre y cuando responda a una necesidad de la entidad. Cabe resaltar que no se pueden pedir artículos por razones estéticas. El proceso para hacer esto consiste en:
\begin{enumerate}
    \item Encontrar el tipo de artículo deseado.
    \item Recolectar información de dicho artículo tanto como se pueda. Esta información incluye, pero no se limita a: 
    \begin{itemize}
        \item Traer un previsto de las medidas del artículo, dónde se ubicará y el rango de tolerancia de dichas medidas.
        \item El peso del artículo.
        \item Si es un producto específico, agregar la marca del producto, quién lo fabrica.
        \item Especificaciones sobre tipo de corriente con la que trabaja caso que sea un aparato electrónico.
    \end{itemize}
\end{enumerate}
Tome en cuenta que esta solicitud de inclusión puede ser declinada por parte de OSUM y por tanto es mejor realizar el proceso caso que se requieran realizar correcciones. Le enfatizo nuevamente que estas solicitudes \textbf{sólo se tramitan cuando responden a una verdadera necesidad} que los artículos dentro del catálogo no pueden suplir.\par 
Cualquier consulta o duda que tengan sobre las instrucciones aquí brindadas o sobre el proceso en general, pueden evacuarla escribiéndome ya sea por correo o por teléfono. Me despido y les deseo muchos éxitos en este nuevo periodo.\par
Muy atentamente;\par
\bigskip
\bigskip
\bigskip
\begin{spacing}{1}
\textit{\textbf{Jonathan Josué Carcache Ríos}}\par
\textit{Coordinación}
\textit{Secretaria de Finanzas}
\end{spacing}
\medskip
\begin{flushleft}\begin{spacing}{1}
 \scriptsize{CC. Archivo/JIRR.

 }
\end{spacing}\end{flushleft}
%%%%%%%%%%%% Contents end %%%%%%%%%%%%%%%%
%Indent whole paragrah: https://tex.stackexchange.com/questions/35933/indenting-a-whole-paragraph
%Dates: https://tex.stackexchange.com/questions/185548/the-year-in-roman-and-the-month-in-text
%SKIPS:https://tex.stackexchange.com/questions/41476/lengths-and-when-to-use-them/41488
%SKIPS2: https://tex.stackexchange.com/questions/476/what-if-anything-is-the-advantage-of-bigskip-and-friends-over-vspace
%%%%%%%%%%%%%%%%%%%%%%%%%%%%%%%%%%%%%%%%%%
%\newpage
%\nocite{*}
%\bibliographystyle{plain}
%\bibliography{bibi}
\end{document} 